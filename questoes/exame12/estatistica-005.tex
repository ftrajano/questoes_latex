Em relação às classificações obtidas num teste de $12^{\circ}$ ano de três turmas - A, B e C - verificou-se que $\overline{x}_A = 14,5$, $s_A = 0,2$, $\overline{x}_B = 13,5$, $s_B = 0,5$ e $\overline{x}_C = 14,5$, $s_C = 0,5$. Podemos afirmar que:
\begin{enumerate}
\item [A)] Na turma A, a distribuição das classificações é mais dispersa do que na turma B.
\item [B)] Na turma B, a distribuição das classificações é mais dispersa do que na turma C.
\item [C)] Na turma C, a distribuição das classificações é mais dispersa do que na turma A.
\item [D)] Se na turma B a professora aumentar um valor a todos os alunos, a distribuição é mais dispersa na turma B do que na turma C.
\end{enumerate}
