Um contrato de empr�stimo prev� que quando uma parcela � paga de forma antecipada, conceder-se-� uma redu��o de juros de acordo com o per�odo de antecipa��o. Nesse caso, paga-se o valor presente, que � o valor, naquele momento, de uma quantia que deveria ser paga em uma data futura. Um valor presente P submetido a juros compostos com taxa $i$, por um per�odo de tempo $n$, produz um valor futuro V determinado pela f�rmula 
\begin{equation*}
V=P.(1+i)^n
\end{equation*}

Em um contrato de empr�stimo com sessenta parcelas fixas mensais, de R\$ 820,00, a uma taxa de juros de 1,32\% ao m�s, junto com a trig�sima parcela ser� paga antecipadamente uma outra parcela, desde que o desconto seja superior a 25\% do valor da parcela. 

Utilize 0,2877 como aproxima��o para  $\ln (\frac 4 3 )$  e 0,0131 como aproxima��o para ln (1,0132).
A primeira das parcelas que poder� ser antecipada junto com a 30� � a 

\begin{enumerate}
\item[a)]56�
\item[b)]55�
\item[c)]52�
\item[d)]51�
\item[e)]45�
\end{enumerate}