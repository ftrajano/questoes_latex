Uma empresa deseja iniciar uma campanha publicit�ria divulgando uma promo��o para seus poss�veis consumidores. Para esse tipo de campanha, os meios mais vi�veis s�o a distribui��o de panfletos na rua e an�ncios na r�dio local. Considera-se que a popula��o alcan�ada pela distribui��o de panfletos seja igual � quantidade de panfletos distribu�dos, enquanto que a alcan�ada por um an�ncio na r�dio seja igual � quantidade de ouvintes desse an�ncio. O custo de cada an�ncio na r�dio � de R$ 120,00, e a estimativa � de que seja ouvido por 1 500 pessoas. J� a produ��o e a distribui��o dos panfletos custam R$ 180,00 cada 1 000 unidades. Considerando que cada pessoa ser� alcan�ada por um �nico desses meios de divulga��o, a empresa pretende investir em ambas as m�dias. 
Considere X e Y os valores (em real) gastos em an�ncios na r�dio e com panfletos, respectivamente. 
O n�mero de pessoas alcan�adas pela campanha ser� dado pela express�o 
\begin{enumerate}
\item[a)]$\frac {50X}{4}+\frac{50Y}{9}$
\item[b)]$\frac {50X}{9}+\frac{50Y}{4}$
\item[c)]$\frac {4X}{50}+\frac{4Y}{50}$
\item[d)]$\frac {50}{4X}+\frac{50}{9Y}$
\item[e)]$\frac {50}{9X}+\frac{50Y}{4Y}$
\end{enumerate}