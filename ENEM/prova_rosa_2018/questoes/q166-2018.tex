Um rapaz estuda em uma escola que fica longe de sua casa, e por isso precisa utilizar o transporte p�blico. Como � muito observador, todos os dias ele anota a hora exata (sem considerar os segundos) em que o �nibus passa pelo ponto de espera. Tamb�m notou que nunca consegue chegar ao ponto de �nibus antes de 6 h 15 min da manh�. Analisando os dados coletados durante o m�s de fevereiro, o qual teve 21 dias letivos, ele concluiu que 6 h 21 min foi o que mais se repetiu, e que a mediana do conjunto de dados � 6 h 22 min. 
A probabilidade de que, em algum dos dias letivos de fevereiro, esse rapaz tenha apanhado o �nibus antes de 6 h 21 min da manh� �, no m�ximo, 
\begin{enumerate}
\item[a)]$\frac 4 {21}$
\item[b)]$\frac 5 {21}$
\item[c)]$\frac 6 {21}$
\item[d)]$\frac 7 {21}$
\item[e)]$\frac 8 {21}$
\end{enumerate}
