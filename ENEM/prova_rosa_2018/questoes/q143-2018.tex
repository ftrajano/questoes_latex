O Sal�o do Autom�vel de S�o Paulo � um evento no qual v�rios fabricantes exp�em seus modelos mais recentes de ve�culos, mostrando, principalmente, suas inova��es em design e tecnologia. 
(Dispon�vel em: http://g1.globo.com. Acesso em: 4 fev. 2015 (adaptado).)
Uma montadora pretende participar desse evento com dois estandes, um na entrada e outro na regi�o central do sal�o, expondo, em cada um deles, um carro compacto e uma caminhonete. 
Para compor os estandes, foram disponibilizados pela montadora quatro carros compactos, de �modelos distintos, e seis caminhonetes de diferentes cores para serem escolhidos aqueles que ser�o expostos. A posi��o dos carros dentro de cada estande � irrelevante. 
Uma express�o que fornece a quantidade de maneiras diferentes que os estandes podem ser compostos � 
\begin{enumerate}
\item[a)]$A^4_{10}$
\item[b)]$C^4_{10}$
\item[c)]$C^2_{4}\times C^2_6\times 2\times2$
\item[d)]$A^2_4\times A^2_6 \times 2\times 2$
\item[e)]$C^2_4\times C^2_6$
\end{enumerate}