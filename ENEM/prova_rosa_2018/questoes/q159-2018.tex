Com o avan�o em ci�ncia da computa��o, estamos pr�ximos do momento em que o n�mero de transistores no processador de um computador pessoal ser� da mesma ordem de grandeza que o n�mero de neur�nios em um c�rebro humano, que � da ordem de 100 bilh�es. 
Uma das grandezas determinantes para o desempenho de um processador � a densidade de transistores, que � o n�mero de transistores por cent�metro quadrado. Em 1986, uma empresa fabricava um processador contendo 100 000 transistores distribu�dos em 0,25 cm2 de �rea. Desde ent�o, o n�mero de transistores por cent�metro quadrado que se pode colocar em um processador dobra a cada dois anos (Lei de Moore). 

Considere 0,30 como aproxima��o para $\log_{10} 2$.
Em que ano a empresa atingiu ou atingir� a densidade de 100 bilh�es de transistores? 
\begin{enumerate}
\item[a)]1999
\item[b)]2002
\item[c)]2022
\item[d)]2026
\item[e)]2146
\end{enumerate}

Texto adaptado: (Dispon�vel em: www.pocket-lint.com. Acesso em: 1 dez. 2017 (adaptado)).
